\documentclass{article}

\usepackage{Sweave}
\begin{document}
\Sconcordance{concordance:barents_SM.tex:barents_SM.Rnw:%
1 14 1 1 0 18 1 1 2 1 0 5 1 3 0 1 2 5 1 1 2 5 0 1 3 4 1 1 3 21 0 1 2 2 %
1 1 3 35 0 1 2 2 1 1 3 21 0 1 2 3 1 1 3 18 0 1 2 4 1 1 2 1 0 3 1 5 0 1 %
1 26 0 1 2 4 1 1 2 6 0 1 1 5 0 1 1 5 0 1 1 6 0 1 2 6 1 1 2 7 0 1 2 2 1 %
1 2 327 0 1 1 12 0 1 2 2 1 1 2 1 0 2 1 108 0 1 2 6 1 1 2 1 0 1 4 3 0 2 %
1 6 0 1 2 5 1 1 2 10 0 1 3 2 1 1 2 1 0 1 1 9 0 1 2 3 1 1 2 1 0 2 1 5 0 %
1 1 4 0 1 2 6 1 1 2 1 0 3 1 6 0 2 2 1 0 1 1 4 0 1 2 6 1 1 2 5 0 1 2 5 1 %
1 2 5 0 1 2 1 1}


\section{R Environment}
A few libraries are to be loaded.

\begin{Schunk}
\begin{Sinput}
> library(RCaN) #the main package
> library(ggplot2) #to draw results
> library(coda) #to explore mcmc 
> library(dplyr) #to manipulate data frame
> library(xtable) #to create latex tables
\end{Sinput}
\end{Schunk}

\section{The RCaN file}
Parameters, observations and constraints have been gathered in an Excel file with a specific structure. 

\begin{Schunk}
\begin{Sinput}
> NAMEFILE <- '/Users/christianmullon/Desktop/Ocean/BarentsSeaReconstructions_01_02_21.xlsx'
> library("xlsx")
\end{Sinput}
\end{Schunk}

\subsection{Components}

\begin{Schunk}
\begin{Sinput}
> COMPONENTS <- read.xlsx(NAMEFILE, 1) 
> head(COMPONENTS)
\end{Sinput}
\begin{Soutput}
                    PathName
1              Caracteristic
2 <NAME VERSION AND AUTHORS>
3                 Model Name
4     Model Version and date
5                    Authors
6         <DOMAIN AND UNITS>
  X.Users.christianmullon.Desktop.Ocean.BarentsSeaReconstructions_01_02_21.xlsx
1                                                                         Value
2                                                                              
3                                              Barents Sea reconstructions 2021
4                                                           2.0 - February 2021
5                                             Benjamin Planque and Elliot Sivel
6                                                                              
                                                                                                 NA.
1                                                                                               <NA>
2                                                                                                   
3                                                                   Title/name of the food-web model
4                                          Specific code used to uniquely identify the model version
5  Names and addresses of the principal investigators associated with the food-web model development
6                                                                                                   
\end{Soutput}
\end{Schunk}


\section{Building polytope}

\begin{Schunk}
\begin{Sinput}
> POLYTOPE <- buildCaN(NAMEFILE)
\end{Sinput}
\end{Schunk}

\section{Checking polytope}

\section{Sampling polytope}

\begin{Schunk}
\begin{Sinput}
> begin = Sys.time()
> SAMPLE <- sampleCaN(POLYTOPE, 
+                       N=100,thin=100, 
+                       nchain=2,
+                       ncore=2)